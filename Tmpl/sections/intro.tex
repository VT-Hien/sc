\section{Introduction}

\subsection{Dysarthria}

\begin{frame}{What is Dysarthria?}
    \begin{itemize}
        \item \textbf{Dysarthria} is a motor speech disorder caused by damage to the nervous system.
        \item It affects the muscles that control speech, leading to slurred or unclear speech.
        \item It does not impact intelligence but can create significant communication challenges.
    \end{itemize}
\end{frame}

\begin{frame}{Symptoms of Dysarthria}
    \begin{itemize}
        \item Slurred speech, weak voice, or inability to control volume.
        \item Speaking too fast or too slow, difficulty regulating speech rhythm.
        \item Difficulty swallowing, increased risk of choking while eating.
    \end{itemize}
   

\end{frame}

\begin{frame}{Prevalence of Dysarthria}
    \begin{itemize}
        \item Around 25\% of stroke patients exhibit symptoms of Dysarthria.
        \item In Parkinson’s disease, about 70\% - 80\% of patients experience speech disorders.
    \end{itemize}
     \begin{table}[]
    \centering
    \begin{tabular}{l c}
        \toprule
        Patient Group & Prevalence of Dysarthria \\
        \midrule
        Stroke & 25\% \\
        Parkinson’s disease & 70\% - 80\% \\
        Traumatic brain injury & 50\% \\
        ALS (Amyotrophic Lateral Sclerosis) & 90\% \\
        Cerebral palsy & 50\% - 60\% \\
        \bottomrule
    \end{tabular}
    \caption{Prevalence of Dysarthria by Patient Group}
\end{table}
\end{frame}

\begin{frame}{Geographical Distribution of Dysarthria}
    \begin{itemize}
        \item Dysarthria affects approximately 2\% - 3\% of the global population.
        \item In developed countries (USA, Canada, Europe), Dysarthria is more common due to longer life expectancy and a higher incidence of neurodegenerative diseases.
        \item In developing countries, Dysarthria is underdiagnosed due to limited healthcare infrastructure.
    \end{itemize}
\end{frame}

\begin{frame}{Diagnosis Methods}
        \begin{itemize}
            \item Speech assessment and neurological examination.
            \item MRI and CT scan to identify brain damage.
        \end{itemize}
\end{frame}
\subsection{MFCC}


\begin{frame}
    \frametitle{What is MFCC?}
    \begin{itemize}
        \item MFCC (Mel-frequency Cepstral Coefficients) is a set of feature coefficients extracted from audio signals.
        \item Based on the Mel scale, mimicking how the human ear perceives sound frequencies.
        \item Applications: Speech recognition, audio classification, emotion analysis.
    \end{itemize}
\end{frame}

\begin{frame}
    \frametitle{MFCC Computation Process - Overview}
    \begin{itemize}
        \item The MFCC extraction process consists of 7 main steps.
        \item Goal: Convert audio signals into compact feature coefficients.
        \item Steps are performed sequentially on each signal frame.
    \end{itemize}
\end{frame}
\begin{frame}
\frametitle{MFCC Computation Process}
    \begin{enumerate}
        \item \textbf{Pre-emphasis:} Enhances high frequencies to balance the spectrum. Formula: \( y(n) = x(n) - 0.97 x(n-1) \).
        \item \textbf{Framing:} Divides the signal into short frames (20-40ms) with overlap.
        \item \textbf{Windowing:} Applies a Hamming window to reduce edge effects.
        \item \textbf{Fast Fourier Transform (FFT):} Converts the signal to the frequency domain, yielding the power spectrum.
    \end{enumerate}
\end{frame}

% Frame 2: Steps 5-7
\begin{frame}
    \frametitle{MFCC Computation Process}
    \begin{enumerate}
        \setcounter{enumi}{4}
        \item \textbf{Mel Filterbank:} Applies triangular filters on the Mel scale: $$ m(f) = 2595 \log_{10}(1 + \frac{f}{700}) $$.
        \item \textbf{Logarithm:} Compresses filterbank energies with $$ \log S_m $$.
        \item \textbf{Discrete Cosine Transform (DCT):} Produces MFCC coefficients, typically keeping 12-13. Formula: \( c_n = \sum \log S_m \cos\left(\frac{\pi n (m-0.5)}{M}\right) \).
    \end{enumerate}
\end{frame}

