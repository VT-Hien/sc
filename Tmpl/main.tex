%%%%%%%%%%%%%%%%%%%%%%%%%%%%%%%%%%%%%%%%%
% Beamer Presentation - LaTeX Template
% Version 2.0 (March 8, 2022)
% Original Template: https://www.LaTeXTemplates.com
% Author: Vel (vel@latextemplates.com)
% License: CC BY-NC-SA 4.0

% Este modelo de apresentação foi 
% criado a partir do modelo de Giovanni Spadaro.
% Disponível em: https://github.com/Giovo17/presentation-template-unict-lm-data
%
% Adaptado por Lucas Amaral Taylor para criar uma versão especial 
% para os alunos de Matemática e Estatística da USP (IME-USP).
% Disponível em: https://github.com/lucasamtaylor01/IME-template
%%%%%%%%%%%%%%%%%%%%%%%%%%%%%%%%%%%%%%%%%

%----------------------------------------------------------------------------------------
% CLASSE DO DOCUMENTO E CONFIGURAÇÕES BÁSICAS
%----------------------------------------------------------------------------------------
\documentclass[
    11pt,               % Tamanho padrão da fonte
    % t,                % Alinhar verticalmente ao topo
    %aspectratio=169,   % Definir proporção 16:9
]{beamer}
\graphicspath{{img/}}         % Define o diretório das imagens

%----------------------------------------------------------------------------------------
% PACOTES NECESSÁRIOS
%----------------------------------------------------------------------------------------
\usepackage{
    booktabs,     % Melhora a aparência das linhas em tabelas
    palatino,     % Define Palatino como fonte principal
    subcaption    % Suporte para subfiguras
}
\usepackage[default]{opensans}  % Define Open Sans como fonte secundária
\input{config/code_langs}       % Importa configurações para highlight de código
\usepackage{xcolor}
\usepackage{tcolorbox}

%----------------------------------------------------------------------------------------
% CONFIGURAÇÃO DO TEMA
%----------------------------------------------------------------------------------------
% Tema Base
\usetheme{Boadilla}                          % Define o tema principal
\useinnertheme{circles}                      % Tema interno com círculos
\useoutertheme{miniframes}                   % Tema externo com miniframes
\setbeamertemplate{navigation symbols}{}     % Remove símbolos de navegação

% Cores Personalizadas
\definecolor{primaryColor}{RGB}{20,45,105}   % Cor primária - azul escuro
\definecolor{secondaryColor}{RGB}{0,100,160} % Cor secundária - azul médio

% Configurações de Cores
\setbeamercolor{structure}{fg=primaryColor}
\setbeamercolor{palette primary}{bg=primaryColor, fg=white}
\setbeamercolor{palette secondary}{bg=secondaryColor, fg=white}
\setbeamercolor{title}{bg=primaryColor, fg=white}

% Cores do Cabeçalho e Rodapé
\setbeamercolor{headline}{bg=secondaryColor, fg=white}
\setbeamercolor{section in head/foot}{bg=primaryColor, fg=white}
\setbeamercolor{subsection in head/foot}{bg=secondaryColor, fg=white}
\setbeamercolor{author in head/foot}{bg=primaryColor, fg=white}
\setbeamercolor{title in head/foot}{bg=secondaryColor, fg=white}
\setbeamercolor{date in head/foot}{bg=primaryColor, fg=white}
\setbeamercolor{page number in head/foot}{bg=primaryColor, fg=white}

%----------------------------------------------------------------------------------------
% BIBLIOGRAFIA
%----------------------------------------------------------------------------------------
\usepackage[style=alphabetic,backend=biber]{biblatex}
\addbibresource{bibliografia.bib}

%----------------------------------------------------------------------------------------
% INFORMAÇÕES DA APRESENTAÇÃO
%----------------------------------------------------------------------------------------
\title[Título]{Dysarthria Classification}          % [Versão curta]{Versão completa}
\author[Nome abreviado]{Vo Tran Hien - 23125056\\ Pham Minh Duc - 20125080\\Vu Mai Thuy - 23125046\\ Nguyen Dang Huu Thinh - 20125071}            % [Versão curta]{Nome completo}
\institute[IME-USP]{Univerisy of Science, Viet Nam National Univerisy, HCMC\\ Institution of Information Technology}
\date[Ano]{MÊS / ANO}

%----------------------------------------------------------------------------------------
% INÍCIO DO DOCUMENTO
%----------------------------------------------------------------------------------------
\begin{document}

% Slide de título com logo
\begin{frame}
    \titlepage
\end{frame}

% Sumário
\begin{frame}
    \frametitle{Dysarthria and Non-Dysarthria Speech Classification}
    \tableofcontents
\end{frame}

% Inclusão das seções
\section{Introduction}

\subsection{Dysarthria}

\begin{frame}{What is Dysarthria?}
    \begin{itemize}
        \item \textbf{Dysarthria} is a motor speech disorder caused by damage to the nervous system.
        \item It affects the muscles that control speech, leading to slurred or unclear speech.
        \item It does not impact intelligence but can create significant communication challenges.
    \end{itemize}
\end{frame}

\begin{frame}{Symptoms of Dysarthria}
    \begin{itemize}
        \item Slurred speech, weak voice, or inability to control volume.
        \item Speaking too fast or too slow, difficulty regulating speech rhythm.
        \item Difficulty swallowing, increased risk of choking while eating.
    \end{itemize}
   

\end{frame}

\begin{frame}{Prevalence of Dysarthria}
    \begin{itemize}
        \item Around 25\% of stroke patients exhibit symptoms of Dysarthria.
        \item In Parkinson’s disease, about 70\% - 80\% of patients experience speech disorders.
    \end{itemize}
     \begin{table}[]
    \centering
    \begin{tabular}{l c}
        \toprule
        Patient Group & Prevalence of Dysarthria \\
        \midrule
        Stroke & 25\% \\
        Parkinson’s disease & 70\% - 80\% \\
        Traumatic brain injury & 50\% \\
        ALS (Amyotrophic Lateral Sclerosis) & 90\% \\
        Cerebral palsy & 50\% - 60\% \\
        \bottomrule
    \end{tabular}
    \caption{Prevalence of Dysarthria by Patient Group}
\end{table}
\end{frame}

\begin{frame}{Geographical Distribution of Dysarthria}
    \begin{itemize}
        \item Dysarthria affects approximately 2\% - 3\% of the global population.
        \item In developed countries (USA, Canada, Europe), Dysarthria is more common due to longer life expectancy and a higher incidence of neurodegenerative diseases.
        \item In developing countries, Dysarthria is underdiagnosed due to limited healthcare infrastructure.
    \end{itemize}
\end{frame}

\begin{frame}{Diagnosis Methods}
        \begin{itemize}
            \item Speech assessment and neurological examination.
            \item MRI and CT scan to identify brain damage.
        \end{itemize}
\end{frame}
\subsection{MFCC}


\begin{frame}
    \frametitle{What is MFCC?}
    \begin{itemize}
        \item MFCC (Mel-frequency Cepstral Coefficients) is a set of feature coefficients extracted from audio signals.
        \item Based on the Mel scale, mimicking how the human ear perceives sound frequencies.
        \item Applications: Speech recognition, audio classification, emotion analysis.
    \end{itemize}
\end{frame}

\begin{frame}
    \frametitle{MFCC Computation Process - Overview}
    \begin{itemize}
        \item The MFCC extraction process consists of 7 main steps.
        \item Goal: Convert audio signals into compact feature coefficients.
        \item Steps are performed sequentially on each signal frame.
    \end{itemize}
\end{frame}
\begin{frame}
\frametitle{MFCC Computation Process}
    \begin{enumerate}
        \item \textbf{Pre-emphasis:} Enhances high frequencies to balance the spectrum. Formula: \( y(n) = x(n) - 0.97 x(n-1) \).
        \item \textbf{Framing:} Divides the signal into short frames (20-40ms) with overlap.
        \item \textbf{Windowing:} Applies a Hamming window to reduce edge effects.
        \item \textbf{Fast Fourier Transform (FFT):} Converts the signal to the frequency domain, yielding the power spectrum.
    \end{enumerate}
\end{frame}

% Frame 2: Steps 5-7
\begin{frame}
    \frametitle{MFCC Computation Process}
    \begin{enumerate}
        \setcounter{enumi}{4}
        \item \textbf{Mel Filterbank:} Applies triangular filters on the Mel scale: $$ m(f) = 2595 \log_{10}(1 + \frac{f}{700}) $$.
        \item \textbf{Logarithm:} Compresses filterbank energies with $$ \log S_m $$.
        \item \textbf{Discrete Cosine Transform (DCT):} Produces MFCC coefficients, typically keeping 12-13. Formula: \( c_n = \sum \log S_m \cos\left(\frac{\pi n (m-0.5)}{M}\right) \).
    \end{enumerate}
\end{frame}


\include{sections/relatedWork}
\include{sections/Proccess}
\section{Experiments}

\subsection{Dataset}
\begin{frame}
    \frametitle{Old Dataset Overview}
    \textbf{Total Participants:} 13  
    \begin{itemize}
        \item Female Without Dysarthria: 2 participants
        \item Female With Dysarthria: 3 participants
        \item Male Without Dysarthria: 4 participants
        \item Male With Dysarthria: 4 participants
    \end{itemize}
    After excluding participants with no valid \texttt{.wav} files, the dataset effectively contains **11 participants**.
\end{frame}

\begin{frame}
    \frametitle{Old Dataset - Female Participants}
    \textbf{Female Without Dysarthria: 2 participants}
    \begin{itemize}
        \item FC01: 1 session
        \item FC03: 3 sessions (1 missing \texttt{.wav} files, effectively 2 sessions)
    \end{itemize}
    
    \textbf{Female With Dysarthria: 3 participants}
    \begin{itemize}
        \item F01: 1 session
        \item F03: 3 sessions
        \item F04: 1 session (no \texttt{.wav} files available)
    \end{itemize}
\end{frame}

\begin{frame}
    \frametitle{Old Dataset - Male Participants}
    \textbf{Male Without Dysarthria: 4 participants}
    \begin{itemize}
        \item MC01: 1 session (no \texttt{.wav} files)
        \item MC02: 1 session
        \item MC03: 1 session (missing \texttt{.txt}, but \texttt{.wav} available)
        \item MC04: 2 sessions
    \end{itemize}
    
    \textbf{Male With Dysarthria: 4 participants}
    \begin{itemize}
        \item M01: 3 sessions
        \item M03: 2 sessions
        \item M04: 2 sessions
        \item M05: 1 session
    \end{itemize}
\end{frame}

% --- NEW DATASET ---

\begin{frame}
    \frametitle{New Dataset Overview}
    \textbf{Total Participants:} 15  
    \textbf{Recording Method:} Array Microphones and Head Microphones  
    The dataset is divided into:
    \begin{itemize}
        \item Female Without Dysarthria: 3 participants
        \item Female With Dysarthria: 3 participants
        \item Male Without Dysarthria: 4 participants
        \item Male With Dysarthria: 5 participants
    \end{itemize}
\end{frame}

\begin{frame}
    \frametitle{New Dataset - Female Participants}
    \textbf{Female Without Dysarthria: 3 participants}
    \begin{itemize}
        \item FC01: (Array Mic: 1 session, Head Mic: 1 session)
        \item FC02: (Array Mic: 2 sessions, Head Mic: 1 session)
        \item FC03: (Array Mic: 3 sessions, Head Mic: 3 sessions)
    \end{itemize}
    
    \textbf{Female With Dysarthria: 3 participants}
    \begin{itemize}
        \item F01: (Array Mic: 1 session, Head Mic: 1 session)
        \item F03: (Array Mic: 3 sessions, Head Mic: 3 sessions)
        \item F04: (Array Mic: 2 sessions, Head Mic: 1 session)
    \end{itemize}
\end{frame}

\begin{frame}
    \frametitle{New Dataset - Male Participants}
    \textbf{Male Without Dysarthria: 4 participants}
    \begin{itemize}
        \item MC01: (Array Mic: 3 sessions, Head Mic: 3 sessions)
        \item MC02: (Array Mic: 2 sessions, Head Mic: 2 sessions)
        \item MC03: (Array Mic: 2 sessions, Head Mic: 2 sessions)
        \item MC04: (Array Mic: 2 sessions, Head Mic: 1 session)
    \end{itemize}
    
    \textbf{Male With Dysarthria: 5 participants}
    \begin{itemize}
        \item M01: (Array Mic: 2 sessions, Head Mic: 2 sessions)
        \item M02: (Array Mic: 2 sessions, Head Mic: 2 sessions)
        \item M03: (Array Mic: 1 session, Head Mic: 1 session)
        \item M04: (Array Mic: 2 sessions, Head Mic: 1 session)
        \item M05: (Array Mic: 1 session, Head Mic: 2 sessions)
    \end{itemize}
\end{frame}

\subsection{Baseline}
\begin{frame}{Frame Title}
    
\end{frame}
\include{sections/dfsf}
\include{sections/section00}
\include{sections/section01}
\include{sections/section02}
\include{sections/section03}

\begin{frame}
    \begin{center}
        {\Huge Fim da apresentação!}
    \end{center}
\end{frame}

\end{document}


